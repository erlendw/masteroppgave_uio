
\documentclass[]{uiophd}
\usepackage[utf8]{inputenc}% or whatever you use
\usepackage{listings}
\usepackage{xcolor}
\usepackage{graphicx}
\usepackage{minted}
\usepackage[T1]{fontenc}
\usepackage[utf8]{inputenc}
\graphicspath{ {./} }
\usepackage{helvet}
\definecolor{codegreen}{rgb}{0,0.6,0}
\definecolor{codegray}{rgb}{0.5,0.5,0.5}
\definecolor{codepurple}{rgb}{0.58,0,0.82}
\definecolor{backcolour}{rgb}{0.95,0.95,0.92}


\lstdefinestyle{mystyle}{
    backgroundcolor=\color{backcolour},   
    commentstyle=\color{codegreen},
    keywordstyle=\color{magenta},
    numberstyle=\tiny\color{codegray},
    stringstyle=\color{codepurple},
    basicstyle=\ttfamily\footnotesize,
    breakatwhitespace=false,         
    breaklines=true,                 
    captionpos=b,                    
    keepspaces=true,                 
    numbers=left,                    
    numbersep=5pt,                  
    showspaces=false,                
    showstringspaces=false,
    showtabs=false,                  
    tabsize=2
}
 
\lstset{style=mystyle}
\lstset{frame=lrb,xleftmargin=\fboxsep,xrightmargin=-\fboxsep}
\begin{document}





\title{\includegraphics[scale=0.5]{toppfelt-english.jpg}\\\\
Using sensors networks and cloud technology to manage water usage in agriculture.
\\\\
\large A technology feasibility study}
\author{Erlend Westbye}


\frontmatter
\maketitle
\tableofcontents
\linespread{1.3}



\mainmatter

\chapter{Introduction}
Strukturen over må selvsagt tilpasses det aktuelle problem og type oppgave, for deg er vel erfaringsdelen viktig, men kan kanskje si det er en slags «evaluation»?
Det kan være ok å sammenligne erfaringene med hva slag forventninger/hypoteser man hadde i utgangspunktet.
Hva var ventet? Hva var overraskende? Noen råd du vil gi ?
\\\\
motivation for the work, description of the problem, 
goal of the thesis (should follow from the motivation), 
approach (how you approached the problem)
research method (optional)
summary of results/contributions
structure of the rest of the thesis (not included)

\section{Why is high water usage a problem}
Many are like me and turn on the shower without giving much thought to the amazing innovation that has gone in to getting the water running through the pipes in to my apartment. But i got to admit, I think less about the water it self. Especially in a country like Norway where water is abundant, and there is no significant visible cost attached to turning on the shower as both water and electricity is cheap here.
\\\\
Although we rarely think about it water is a finite resource. Drinkable water makes up [1] of the world's water supply. This means that we [might/probably/are] heading in to a future where the seemingly endless access might be dwindling down due to pollution, drought and heavy agricultural needs water might just become the new gold. We will not be arguing whether the commodification of water is the worst thing that can happen to modern society, whether the free market knows what is right for the world, or how much of a toll agriculture is taking on the atmosphere of the earth. Instead we will be taking a long look in to the feasibility of using modern technology to reduce water usage in one of the most water intensive industries; Agriculture.
\\\\
Water is high on the agenda. The united nations has a list of sustainable development goals, where water is one of the items on the list. The OECD has also put forth water as one of the important things to put on the agenda. In a 2018 note they also mentioned the gathering of information as one of the important steps in improving the situation 
\\\\
"Improve information systems on surface and groundwater quality and flows, help to assess risks, and implement programs tailored to specific challenges."
\\\\
Norges Bank Investment Management (Oljefondet) has also made sustainable water management one of their goals for sustainable investing. 
\\\\
"Companies should recognise the business’
water impact, commit to sustainable water
management and, as relevant, have a clear
water management strategy."

\section{Water around the world}
Excessive water usage can lead to a whole host of issues, especially in countries that are prone to having long drought seasons. Most recently [spring/winter 2018] the [China/South African] capital has been struck by severe water shortages and are facing strict water rations. During the summer of 2018 Norway, as well as a lot of Europe experienced severe droughts. And a lot of farmers had to start feeding livestock with winter feed as early as {june, july}. So even water rich Norway has seen that we are not immune to the changing climate and the effect that water shortages can have on our domestic supply of agricultural goods.
\\\\
This is an example of one of the scenarios that makes water so valuable. The future value of water might be one of the reasons why there has been heavy investing in water and water related industries by some of the leading food and beverage companies in the world.

\section{The commodification of water}
Water is also turning into a very valuable commodity. Although we aren’t seeing the same financial instruments for water as we see for traditional commodities yet, there are economists theorizing that we might be seeing environmental derivatives (finacial products) in the future. And that we might have trading markets for water more like what we are used to seeing with for example oil. There are a lot of interest around controlling central water supplies and the rights to water springs are being aggressively bought up by companies like (nestle, , , ) all over the world.

\section{Ground water depletion}
Groundwater depletion is one of the more severe consequences of overuse of water, and according to [reference, blue gold] the most common cause of groundwater depletion is irrigation. One of the more severe consequences of groundwater depletion is that it can cause sinkholes, which is when the natural integrity of soil/ground is altered and the ground collapses. This is a severe problem in some of the us states that we are going to be writing a bit about in the section on where it is important to reduce water usage, and [insert places].

\section{Questions to be answered by this text}

\subsection{Is cloud technology mature enough for large scale streaming}
The biggest cloud vendor in the world was started in 2006. In the last 13 years the rate of innovation in this space has been staggering. This is something we have seen in the time where we researched and wrote this paper. When it comes to streaming large amounts of data we know that there are many vendors that have achieved this and there are many services for gathering, processing and storing large data sets
\\\\
Examples of massive data stream systems can be found as an example within the financial space. Where companies like OneTick can offer solutuions that can stream and persist every single trade that takes place on an exchange. This gives a good idea of the big data capabilities that can be implemented in cloud. Cloud vendors provide services like hosted data bases, data streaming platforms and data lakes. Developers are getting used to having immense computing capabilities at their finger tips, and data streaming is one of the things that can be implemented in cloud. We will look in to data consistency accross regions, and challanges related to pricing and speed.
\subsection{Is agriculture primed for digitization}
Farming is a part of human evolution and is one of the worlds oldest professions. We where at one point hunter gathererers and as a part of our evolution as a society we learned to domesticate plants and animals. The farming tradition in Norway is rich, and driving through the Norwegian country side shows the large focus we have had on farming as a nation. Farming is an area that contains some incredible innovation on many scientific fronts, we have modified the genes of plants to make them more drought restistant and resistant to pestesides. I would therefore argue that agriculture is an industry where technological innovation is engrained in the activities that farmers take on every day.
\\\\
Farming equipment is going smart, although we have not seen a lot of coverage of this in norway, we have seen international media coverege on farmers going so far as hacking their farming equipment. I would therefore argue that agriculture is an industry that would be happy to adopt new technology as we have seen them do many times over.

\subsection{Can IoT and cloud reduce water usage in agriculture}
This is the central question of what we explored, and we started out with the standpoint that from a intuitive point this is the case. The research paper presented by [vannbruk handler om systemer] gave us a clear overview that in broad strokes water consumption is more of a management problem than a mechanical problem. Some interesting findings of the paper that helped build up our theory is the fact that the age of the farmer had more to say for the water consumption of a farm than the irrigation system suggesting that through "smart" management and digitization of this industry can have a big impact on this industy. Mismanagement of water can lead to a host of problems like we have seen in the introduction, and water management in agriculture is an area we truely believe that information technology can be used for good.

\subsection{Approach}
We chose an exploratory approach where we wanted to gain insight in to what factors we could monitor and how we could architect a system that could support farmers in making good water management decisions. We chose the primary strategy of building out a proof of concept ourselves as well as researching the topic through exploring previous work on the subject. We even went to meet with a farmer to talk about monitoring and what factors are of interest to them. As well as meeting with a leading Norwegian manufactoring automation firm to learn about their sensor offerings as well as how they develop their monitoring platform.
\\\\
With this mixed approach of learning by doing and reading about the subject at hand we have become well equiped to talk about the challanges and problems to be solved to improve water management in agriculture. And how iot and clout technologies can help in improving the water usage in this sector.

\section{The possibilities opened by technology}



This section will cover "IT systems in agriculture"

\section{Research method}
There has been some research in to automated irrigation in the past, and most of the articles that have been published touches on some of the same subjects that we will be exploring in this paper. We will conduct a feasibility study considering the cost of a rudimentary set up set up in both cloud and on premise. We have built out our own monitoring system with cheap and commonly available hardware. We have also written some custom software to get a grip on the amount of data a system like this would produce on a larger scale. Taking in to consideration what data is relevant, and how to gather data from multiple sources. We will look at the cost and technology considerations of this system. Other important factors is the importance of data accuracy and data integrity. Substantial work has gone in to laying the foundation of a reference architecture. We implemented a large portion of the system that is discussed in this paper to learn about the networking and hardware challanges that comes along with this type of distributed wireless sensor network. As a part of this we also tested different cloud plattforms to provide insight in to the pricing and resource consumption in the cloud. 
\\\\
In the end we found limited value in the actual data we collected, but in so far as answering the question of wether or not this system is feasable it helped build our conclusion. The value came primarily from the visualization platform that was built as a part of the reference implementation of this system. This yielded a clear picture of the state of our monitored plants, and gave clear insight that could potentially help with watering in a large scale agricultural set up of a similar setup.

\section{Data collection}
All the data referenced in this paper is obtained by running a version of the system outlined in the proposed architecture section of the paper. The data was collected from [aug 17 to dec 17], and has been run through simple analysis methods to gain the insight presented in this paper.  Other referenced data is collected from publicly available datasets obtained through the referenced materiel
\\\\
The data collected was initially limited to only moisture, this was interesting as we could see the plants consumption over time. One of the challenging parts that we spent a lot of time on was figuring out what thresholds should be set for watering. We quickly relized that the system we had envisioned was ignoring factors like evaporation and we explored the possibility of also monitoring the temperature of the soil in addition to the moisture level. This data proved very interesting as we could se a clear correlation between temperature and the rate of evaporation of water from the soil that we where monitoring.
\\\\
Ultimately we relized that we had taken an oversimplified view of what is important to look at when you try to understand the complex picture of plant health. But we also relize that water is the primary factor that we would like to monitor, and we saw that the data we where able to capture gave a good picture of the water consumption and we do theorize that this type of data can do a lot to cut down on water usage in agriculture.

\section{Research method cont.}
Setting up the sensor networks is a challenging task as you are doing the work of a programmer and a product developer at the same time. What we want to prove is that it is feasible to build this type of system yourself and that there is room for an open source alternative to the commercial products that we will discuss in this paper. We sought out to learn about our subject by actually building a function prototype. To the extent that we felt comfortable talking about the feasibility of this type of system based on what we have learned. It is one thing to read about prior experiments, but a whole other to actually build it out yourself.
\\\\
We also read and gathered information about similar systems as well as spending time looking at the state of the art open source and commercial products in this space. It is also important to mention the cloud aspect of this project as it is one of the key cost drivers. We have explored the use of the vendors Microsoft, Amazon and digital ocean. As well as doing a theoretical comparison of the deployment of the system on physical hardware
\\\\
A part of the research is looking at the comparable literature and comparing and trying to learn from similar projects. We have been especially interested in research from countries that are more drought prone as this type of system potentially could have a big impact there.

\section{Central findings}
\subsection{software}
Through this exercise in exploring the feasability of a large scale agricultural sensor networks we have discovered that this type of software can be implemented by developers with modest professional experience using open source frameworks and public information. The implementation overhead can be kept low as a lot of the software we built is based on open source implementation on different systems. As we se with a lot of systems development, the code-foundation is out there in the public via open source, and it is possible to combine in to new products in a relatively short timeframe.
\subsection{hardware}
Although not very costly the power constraints became a huge roadblock for a proper battery powered wireless sensor network. We found that most of the system could be implemented using commodity hardware bought from cheap sources. But when it comes to managing battery power and implementing a system using stripped down versions of the esp8266 the challenges are no longer trivial to a developer with basic knowledge of electrical engineering and move over to a realm o physical hardware manufacturing.
\subsection{cloud}
We tested three different cloud vendors and found them all to be more that adequate for our purposes. Our primary implementation used a common linux distro (ubuntu), mongodb installed on said server on the platfrom Digital ocean. In later times we have also explored hosted services like RDS from AWS and documentdb from Azure. We found that the hosted services from the likes of microsft and amazon had a big impact on the cost of implementing this type of system.
\subsection{cost}
We where able to run a basic implementation for a low cost, and have kept it running at approximately 5 usd per month. The hardware is also relatively cheap depending on the type of sensor used. With the most basic sensor dedscribed in this paper we where able to keep the cost to approximately 10 usd per mote, and with the more advanced sensor described later we where able to do it at about 25 usd. Ordering bulk quantities of the same hardware would yield a substantially lower price.
\subsection{feasability}
We can state for certain that from a technology standpoint this type of system is very possible to build and can be impolemented without excessive technical compitanse, and without high development and hardware costs. From a water use percpective the answer is more dubious but we can say that based on the theory that water usage is tightly coupled to management routines we can argue that this would be of great help to farmers that want to monitor and improve their water usage.


\chapter{Background}

Knowledge that is needed for the reader to understand the thesis describing relevant mwethods, tools, techniques that can not be assumed to be common knowledge
Description of (some) related work and its limitations wrt to the goal of the thesis

\section{Water management in agriculture}
Humans have cultivated crops for a long time, and depending on where you read we see numbers going as far back as 20000 years (The Origin of Cultivation and Proto-Weeds, Long Before Neolithic Farming). And there is a lot of focus on water and watering in many societies around the world. Look at the flood irrigation methods used for rice fields, or the rich cultures that cropped up around the Nile. Watering is something that we humans take very seriously. But as with any endeavour worth doing, it's worth doing right! 
\\\\
We have chosen to take the following approach when looking at excessive irrigation. And we follow the definition of (Estimation of excess water use in irrigated agriculture: A Data Envelopment Analysis approach) "The application of more water than a crop can use". Following is a definition of presition farming "Precision farming explains yield variability at the field level as a function of variability in inputs, including irrigation water."<- this is a note
\\\\

The application of more water than a crop can use, or overwatering, is usually the result of lack of knowledge about soil water content or crop water demand (ET) (from same article) highlights the fact that knowing the moisture content of the soil could be integral in reducing the over-irrigation of crops. Quantification of the extent of overwatering has been shown to be very valuable. (quote) Further supports the claim that quantifying the soil moisture content would be a valuable tool in reducing overwatering in agriculture
\\\\
Quantification of the extent of overwatering has been shown to be very valuable.

\section{Small scale WSN}
\subsection{Sensor nodes}
\subsubsection{NodeMcu}
The sensor nodes are based around the NodeMcu development board. The NodeMcu board is again build around the ESP8366 wifi chip. To develop an application for a NodeMcu board, you must first create the firmware for the ESP8366 chip. The chip comes with standard firmware, but for software development you need more libraries than the standard firmware. The firmware is modular and you can put together your own custom firmware using an online firmware build service called nodemcu-build.com. When you have flashed the firmware, you can write Lua scripts which you again flash to the chip. For our application, only very simple scripts are needed. For our nodes we have scripts which each 30 minutes reads sensor data from the analog and digital input pins on the board, creates a datapoint and a HTTP request and sends the data to the server. When the request is sent, the device is set to sleep mode to save battery/power. The boards needs 3v power to run and can use both batteries and normal power.
\\\\
It is hard to run this type of board off of battery as there are quite a lot of componennts that consume a lot of power. The esp8266 controller has a built in low power sleep mode which can be utilized for ultra low power operation. The problem for our poc is that the board that the esp8266 controller has a lot of additional components like the usb contoller mentioned above. This halted our progress for creating long running nodes in our system. But we managed to have it running for up to five days off of battery power with the nodemcu board. Proving that creating a completely wireless system is possible. We purchased the esp8266 board without the additional components of the nodemcu board, but completing the electrical component work prooved to be more challanging than we first thought. And we decided on leaving that part of the project behind to focus more on the sofrware and proposed architechture of this type of system.

\subsubsection{Sensors}
The nodes record data from a digital temperature sensor and an analog soil moisture sensor. The sensors are read by accessing the digital and analog input pins on the NodeMcu board to which the different sensors are connected. The soil moisture sensor measures the soil resistivity. The soil resistivity is the measurement of how much resistivity an electrical signal experiences travelling through the earth. The less resistance, the more moisture there is in the soil. The temperature is measured with a Sensorion SHT11, an integrated circuit which contains a silicon band-gap sensor.

\subsubsection{Creating datasets}
Data visualization is an important step in showing the power of the data that we have gathered in the proof of concept we did. It illustrates that we are able to gather data the gives a good representation of the current state of a plant/patch of land. As you can see from the illustration underneath this type of data is very intuitive as evaporation is something that we are familiar with
\\\\
For our proof of concept we used two differnt soil moisture sensors. One is a very rudimentary sensor that measures the resistance in the soil between two metal rods. This gives a good indication of moisture level. Water is a good conductor of electricity which means that a reduction in resistance means that there is more water present in the soil
\\\\
The second sensor that we implemented is a humidity sensor paired with a special housing that allows for air to pass through, but not water. This means that as the soil moisture increases the moisture of the air that passes through the sensor. This type of sensor also contains a temperature sensor, which gives us valuable information of the current state of the soil that we are monitoring.
\\\\
We did not measure the sensitivity of our sensors, but to get a usable humidity readout we created a span between the readout from 100 percent humidity (sensor submerged in water) and as close to completely dry soil. For the humidity style sensor we got our baseline verification by checking that the sensor was within a few percentage points of the humidity reported by "metrologisk institutt" outdoor in the area that we where in.
\\\\
The resistance based sensor is analogue, and it is easy to read the information from the sensor. As it is connected to a pin on the board that is able to read resistance. The humidity sensor is digital and required that a library for communicating with the sensor was loaded on to the microcontroller. This type of open source software is often easy to find online for most types of microcontrollers. This also held true for the microcontroller that we where using (nodemcu). As we found a lua implementation that gave a good read out to establish a baseline for temperature and humidity

\section{Simulating WSN}
Our solution uses a python script to simulate a large wireless sensor network(WSN). The script can simulate a data stream from variable sized WSNs with any network topology. For our solution, we chose to simulate a large WSN where the individual motes communicate directly with the main application. Our simulated WSN was organized into virtual physical areas, each area containing 50 motes. This way we can easily scale the simulation for stress testing by changing the number of areas. It’s important to note that a really accurate simulation of a WSN not possible with a script like this as it can’t emit all HTML requests simultaneously. However, taking into account the possible differences in the motes like clocking differences we can assume that it will cause a similar load to the the web application as our simulation.
\\\\
Our simulation was used to both simulate normal WSN behaviour and to stress test the web application. For normal behaviour we simulated a WSN with 500 motes, each emitting momentary temperature and moisture data every 30 minutes. For stress testing we had the script constantly emitting data points from all motes. These are some notes on what we experienced:
\\\\
Normal WSN behaviour: In normal behaviour 500 motes are emitting momentary temperature and moisture data every 30 minutes. This means the web application has to handle 500 HTML post requests each 30 minutes. The web application had no problems handling this kind of traffic.
\\\\
Stress test:  For stress testing we had the script constantly emitting data points from all motes. The main thing we wanted to confirm with a stress test like this was that the hosting service could handle that much traffic without affecting the performance of the web application. Keep in mind that we used the lowest spec’d virtual machines from one of the cheapest IaaS providers. If this kind of hosting can handle traffic from a large WSN, it will virtually remove the cost of hosting a the web application for a WSN like this. 
\\\\
The application had no problems handling the traffic from our stress test. We expected some reduction in the responsiveness, but could not detect any. 


\section{Web Application}
The web application is a quite simple application built on the MEAN-Stack. The application receives and stores the data points from the WSN motes while providing a web interface. The web interface shows one graph for each area in the WSN. The graphs show the average temperature and soil moisture in that area for the last 24 hours. Each point on the graph is an calculation of the average temperature or soil moisture from all the motes. The interface is chosen not for actual use but as a proof that the web application can handle a certain traffic and functionality. 
\\\\
\subsubsection{MEAN-stack}
The MEAN-stack is a combination of technologies that allows developers to create lightweight web application written in all javascript. It consists of the four elements Mongo DB, Express.js, AngularJS and Node.js. MongoDB is a noSQL database, Express.js is a server side javascript framework, AngularJS is a frontend javascript framework and Node.js is a server side javascript runtime environment. 
\\\\
\subsubsection{Database}
For our proof of concept we decided to go with a nosql system. This is different from a regular relational database as the stored files are can handle nested object, instead of using a key relation as in a relational database. This type of system lends itself well to nested objects, as there is a clear relationship and tree like structure to the layout of the data model. Our data model was only concerned with storing the moisture data, and not the kind of administrative data that you might need for this type of system in the real world
\\\\
All calculations and data aggregation are done in the browser. We do this to offload the server, as we want the server to only handle the HTML requests. The performance of the browser side application is decent, however the Javascript library used to generate the graphs have a lot of functionality, like animation that are not really necessary for this purpose, which gives a sub optimal rendering time. For this situation, a simpler library would be better, however the point of the web application is not to prove response time of javascript libraries, the purpose is to prove the responsiveness of the server. We also want to point out that using AngularJS as the main front end framework is not the best fit for this application. For rendering of this kind of front end components(graphs) React in combination with other libraries would be a better choice. 
\subsubsection{hosting/Cloud}
Cloud is just somebody else computer. Whether you are paying a dedicated company or running on a shared platform, in essence it is just somebody elses computer. There is not necessarily dedicated hardware running for your server, but a virtualized computer. For this proof of concept we have looked at the following vendors: Aws, azure and digitalocean. The two former are the hosting giants Amazon and Microsofts offerings and is what is currently commercially viable for larger companies. And digitalocean is a bare bones offering that is geared more towards the customers who are fine doing their own patching and upgrading. The three platforms also differ a lot in that aws and azure offers a large range of software as a service, and digitalocean only offers infrastructure as a service. We will also compare what offerings the vendors have for api gateways, storage {add more}
\\\\
We are hosting our application on a Digital Ocean Droplet. Digital Ocean is a cloud infrastructure company that offers affordable cloud services. A Droplet is a virtual machine instance with varying degrees of computing power, storage and networking functionality. We are hosting our application on the cheapest possible option. We use one Droplet with 1GB memory and 25GB storage. This is costing us 5 dollars per month, which is an insignificant expense. 
\\\\
Often there is no need for more than a simple server to run this type of application. But we also made an effort to explore alternatives when it comes to cloud hosting. We came at this proof of concept with cost as one of the key constraints that we have kept in mind. As this type of system would need to run at a as low as possible cost so that you could see a return on the effort as fast as possible. Seeing how we can get away with five us dollars per month, makes it hard to compete for the other providers in our case. But there it is still relevant as we need to take in to account the maintainability and scalability of our solution.

\section{An Introduction to central themes of research}
The Future Internet is pushing the world towards a state of ubiquitous computing, connectivity and sensing. The concept of Internet of Things(IoT) plays a big part in this paradigm shift. IoT explains the shift in the hierarchy of the Internet where the number of "thing" entities outnumber the number of human entities and makes humans the minority of generators and receivers of traffic. IoT also calls for an increase in Machine to Machine(M2M) communication. One of the most important technologies of the IoT is Wireless Sensor Networks(WSN). A WSN is ''a network of battery-powered sensors interconnected through wireless medium and is typically deployed to serve a specific application purpose''\cite{Ojha2015662}. To fully benefit from the ubiquitous connectivity of the IoT computation power, storage and networking infrastructure must enter the equation. Cloud Computing offers these capabilities and the utilization IoT in conjunction with Cloud Computing is the direction the Future Internet is pushing. This combination of technologies can be applied in a infinite number scenarios for better monitoring, data collection, analytics, forecasting and decision making. Developments in technology and in the prices of Cloud services means that these technologies are now available to almost any business that can make use of them. Agriculture is an area where the potential is tremendous for these technologies to work together, especially from an economic and environmental stand point. Agriculture is an area which highly relies on the climate in which it operates and must therefor adjust to the shifting nature of that climate. Knowledge of this shifting nature is key for successful agriculture. Environmental monitoring, data collection, analytic, forecasting, automated systems for irrigation or fertilization, plant health monitoring and frost detection are all scenarios that apply for both conventional and urban agriculture where these technologies can supply crucial information for important decision making. In this article we will explore the state of the art of these technologies and how they can be applied in different agricultural scenarios. We start by exploring concepts like Cloud Computing and IoT before move on to specific technologies like WSNs and application level protocols and conclude the text with examples of these technologies' applications in agriculture.

\section{Is over watering a management or a mechanical problem}

It is also relevant to provide some context on the common types of irrigation systems that are deployed. These will wary across geographical locations, but it is from a intuitive standpoint fair to assume that irrigation is being done in a similar fashion across the world. I have decided to include the systems discussed in the paper [] as well as describing some of the  other common irrigation methods that we find. The two main differences are: Using rainwater and actual irrigation. \\\\
Rainwater irrigation is an important part of farming and following the weather has a lot of importance in farming. As an example drought can decimate the yield of a crop. We saw this happen in 2018 with livestock feeds like grass here in Norway. In these situation water management becomes especially challenging as wells and other natural sources of water can get depleted. Rain-fed farming is the natural application of water to the soil through direct rainfall. Relying on rainfall is less likely to result in contamination of food products but is open to water shortages when rainfall is reduced.<- fra https://www.cdc.gov/healthywater/other/agricultural/types.html
\\\\
Irrigation systems include but are not limited to surface irrigation, sprinklers, drip irrigation, center pivot irrigation. The main differences for the types of irrigation is the precision of the irrigation, the cost and the labour involved in doing the irrigation activities. Intuitively we would think that there is a high correlation between the type of irrigation used and the amounts of water used. As an example you could think that any irrigation that involves flooding would result in exessive irrigation. But as argued by (Estimation of excess water use in irrigated agriculture: Data Envelopment Analysis approach) A major finding of the study is that there is only a weak relationship between irrigation system type and the level of excess irrigation water used. In particular, flood irrigation <-quote \\\\
This is interesting to our research case as the implication would be that the resource intensive and costly exercise of changing or replacing the irrigation system used would not automatically result in a reduction in water usage. This in our mind highlights that gaining insight in the actual water amount requirements over time could be very valuable input to the management process of a farm. One of  the challenges highlighted by the paper [] is that the gathering of agricultural data is very costly and inaccessible to a large portion of farmers. Given that this paper is already 10 years old, we would argue that the implementation we have demonstrated as working has brought the price of gathering this type of data by what we assume to be a lot. systems are not clearly inefficient and management may play a significant role in the levels of water use efficiency that can be reached.(Estimation of excess water use in irrigated agriculture:
A Data Envelopment Analysis approach)
\\\\
The conclusion that the management plays a significant role in excessive water usage is especially interesting given that the primary goal of this project is to conclude whether or not a system for collecting and visualising water consumption using wsn and cloud is feasible. This conclusion by [] is a big motivating factor and gives a good foundation to discuss the feasibility and value of a system like the one we will lay out in the implementation and design portion of this text.

\section{Cloud Computing}
''Cloud computing is a model for enabling ubiquitous, convenient, on-demand network access to a shared pool of configurable computing resources (e.g., networks, servers, storage, applications, and services) that can be rapidly provisioned and released with minimal management effort or service provider interaction.''\cite{Mell:2011:SND:2206223} This is the National Institute of Standards and Technology's(NIST) definition of Cloud Computing. NIST further define the concept with five characteristics, three service models, and four deployment models\cite{Mell:2011:SND:2206223}. The five characteristics are:
\begin{itemize}
\item \textbf{On-demand self-service} A customer can provision computing capabilities at any time without any human interaction with the vendor.
\item \textbf{Broad network access.} Resources should be available for access from a wide range of devices.
\item \textbf{Resource pooling} The vendors resources should be pooled and served to customers based on demand.
\item \textbf{Rapid elasticity} Computing capabilities should be elastically provisioned to provide rapid scaling 
\item \textbf{Measured service} Resource usage should be monitored, controlled and reported providing transparency for both the vendor and the customer.
\end{itemize}

There are a multitude of different service models, but some of the prominent ones are the once defined by NIST:
\begin{itemize}
\item \textbf{Infrastructure as a Service(IaaS)} ''Provision of processing, storage, networks, and other fundamental computing resources where the consumer is able to deploy and run arbitrary software, which can include operating systems and applications. The consumer does not manage or control the underlying cloud infrastructure but has control over operating systems, storage, and deployed applications.''\cite{Mell:2011:SND:2206223}
\item \textbf{Platform as a Service(PaaS)} A platform for application deployment giving the user the ability to deploy applications created using programming languages, libraries, services, and tools supported by the provider.
\item \textbf{Software as a Service(SaaS)} A software provided to the customer running on the vendors infrastructure. The application should be available from various devices.
\end{itemize}

The 4 deployment models are:
\begin{itemize}
\item \textbf{Public} This is a cloud that is operated by a trusted third party that handles all of the infrastructure and networking for the user. This means that all of the hardware is managed by the trusted third party.
\item \textbf{Private} This is a cloud that is hosted on hardware owned by a company or individual. Here the user would have full control of the hardware. Cloud does not imply that the servers are hosted off site, but these type of clouds are typically hosted in private data centers.  
\item \textbf{Community} A Cloud exclusive for a specific community of consumers. This can be organizations with shared concerns. It can be hosted by any of the organizations or a third party. 
\item \textbf{Hybrid} Here a mix of private and public cloud is used. This is a very common approach as it offers great flexibility. And is widely used as companies still have a lot of capital invested in hardware.
\end{itemize}
 
One of the key technologies that has made the modern cloud possible is virtualization. Computer virtualization refers to to practice of running operating systems on virtual hardware. This is achieved by splitting the resources of physical hardware between different instances of operating systems. This is done by installing software known as a hypervisor on the hardware instead of an operating system like linux on a computer. The hypervisor allows for operating systems to be installed on top of the hypervisor, so that one machine or pool of resources can run many operating systems at the same time.\cite{1430631}


\section{Internet of Things}
IoT is the concept of connecting “things” to the Internet to gather data and perform actions. Examples of uses for IoT range from manufacturing to agriculture. And the concept is often looked at as a way to connect everything from fridges to juice presses to the Internet. The recent development that has made IoT more available is the low price and number of of WiFi, Bluetooth and radio enabled microcontrollers.  This means that any sensors with a compatible interface can be connected to the Internet at a very low cost.
\\\\
The high volume and low time to market of many IoT products has presented a range of challenges like privacy and network security. There has already been large scale issues with connected appliances. The hacking of the connected light bulb Phillips hue is explained in this article \cite{6997469} Despite these concerns the number of IoT devices is rapidly increasing. And we are seeing wide spread adoption in both industrial and consumer scenarios. 

\section{Wireless Sensor Networks}
Wireless Sensor Networks(WSNs) are ''a network of battery-powered sensors interconnected through wireless medium and are typically deployed to serve a specific application purpose'' \cite{Ojha2015662}. This type of network are applied for a variety of different purposes, including environmental monitoring, machine monitoring, safety management, smart buildings and many more. For any application this type of network would be comprised of a variety of sensors, such as temperature, humidity and volatile compound detection sensors \cite{s90604728}. Technological advances have made small, low cost, low power and highly customizable devices commercially available and has improved the viability of large scale sensing networks with a large number of intelligent sensor nodes. A sensor node, nicknamed “mote”, in a WSN typically equipped with one or more sensors, a sensor interface, processing units, transceiver unit and power supply \cite{Gubbi20131645}.
\\\\
For many years the trend in WSN topology has been a network comprised of several motes and one or several gateway nodes. In this type of network the motes communicate with each other and the gateway node using a radio frequency communication technology like ZigBee or Bluetooth. The gateway node can have several functions including computation and Internet connectivity. The motes in this type of network can not communicate with the Internet directly and relies on the gateway nodes for this, making the network a closed or proprietary system with limited communication to the external world. 
\\\\
Current trends are pushing WSNs in the direction of the Internet of Things(IoT) using the Internet Protocol(IP) to connect every mote to the Internet \cite{6064380}. Connecting individual motes to the Internet using IEEE 802.11(WiFi) enables the utilization of IoT platforms and other resources in Cloud services. In some cases applying a gateway based architecture to a IP-enabled, WiFi interconnected WSN can be the natural choice, especially if the gateway node(s) are able to perform challenging computing tasks. This architecture applies the concepts of Fog Computing.
\\\\
Fog Computing brings the resources near the underlying network \cite{6984239}. Compared to the Cloud paradigm's centralized computing, storage and networking resources Fog Computing pushes these resources closer to the edge of the network, closer to the user. For a system that demand both low latency and extensive computation, a Fog architecture would be a natural choice. In context of WSNs this means utilizing a Smart Gateway \cite{6984239}. As explained in \cite{69842392} a Smart Gateway should be ''collecting the data and performing preprocessing, filtering the data and reconstructing it into more useful form, uploading only necessary data to the cloud, keeping check on IoT objects and sensors’ activities, keeping check on energy consumption of power constrained nodes of IoTs, security and privacy of the data, and overall service monitoring and management''. 

\section{Integrating WSNs with Cloud services}
There are two main approaches to the integration of WSN data with a Cloud service. Which one suits your network depends on the architecture of your WSN.
\\\\
All large Cloud providers offer an IoT platform of some kind. Amazon's "AWS IoT", Microsoft's "Azure IoT Suite"(and the list goes on) both offers connectivity for most devices, graphic remote resource monitoring, data storage and analytics, data stream analytics, security and application platforms. These platforms offers a broad array of elastic resources and can be customized to accommodate a WSN. There are also more specialized services like Thingspeak, Xively, Carriots and Kaa aimed only at IoT and might not offer the same kind of underlying infrastructure as the big Cloud vendors.
\\\\
It is also possible to create a more ad-hoc solution using the resources of the Cloud. Using IaaS services from a Cloud provider it is possible to create a service with the specific resources the network needs. K. Lee \textit{et al}. \cite{5678063} explores a Cloud service for WSNs built on Amazon EC2 virtual machines. The article demonstrates ''how sensor networks can be combined with Cloud Computing to allow the offloading of resource-intensive tasks to the Cloud.''\cite{56780637}.

\section{Application layer protocols}
One of the key issues in IoT and WSNs is power efficiency in devices, given that a large scale deployment would be severely compromised if the batteries of the nodes in the network had to be replaced often. Application layer protocols has a big impact on power consumption as communication is the most power hungry task the devices perform. One must also keep in mind that packet loss can effect real time data analytics. These are issues to consider when deciding on a protocol. In this section we discuss the most common application layer protocols for IoT and WSNs.

\subsection{HTTP}
 
''The Hypertext Transfer Protocol (HTTP) is an application-level protocol for distributed, collaborative, hypermedia information systems''\cite{HTTP1996}. HTTP is the dominating protocol for client-server application level communication in the Cloud context. However, HTTP has proven to be energy ineffective in constrained environments, in particular with the small frame sizes and the lossy links of low-power wireless communication that characterizes traffic from IoT devices and sensor nodes\cite{karagiannis2015survey}\cite{7030106}. Therefore, HTTP is seldom used in IoT and WSNs and the following protocols are preferable.

\subsection{CoAP}
The Constrained Application Protocol (CoAP) is ''a specialized web transfer protocol for use with constrained nodes and constrained (e.g., low-power, lossy) networks'' \cite{rfc7252}. It was designed by the Internet Engineering Task Force (IETF) especially for devices with constrained resources. This makes CoAP a good option to use with IoT devices as they are usually resource-constrained. IETF state that ''The nodes often have 8-bit microcontrollers with small amounts of ROM and RAM, while constrained networks such as IPv6 over Low-Power Wireless Personal Area Networks (6LoWPANs) often have high packet error rates and a typical throughput of 10s of kbit/s.  The protocol is designed for machine-to-machine (M2M) applications such as smart energy and building automation.''\cite{rfc7252}. CoAP runs over UDP in order to remove the TCP overhead and to reduce bandwidth requirements. The protocol enables RESTful communication in a client-server architecture with the well-known methods GET, PUT, POST and DELETE and can provide both synchronous and asynchronous responses creating painless interactions with HTTP in web applications.\cite{rfc7252}\cite{karagiannis2015survey}\cite{7030106}.

\subsection{MQTT}
Message Queue Telemetry Transport (MQTT) is a client server publish/subscribe protocol created by IBM with M2M communication in constrained environments in mind. It runs on TCP and is asynchronous. IoT devices are usually resource constrained, making MQTT a good option to use with IoT devices. The protocols publish/subscribe configuration demands less resources than request/response as clients do not have to request updates which decreases the network bandwidth and the need for computational resources. Comparing MQTT to CoAP we see that the UDP-based CoAP produces lower overhead than the MQTT which runs over TCP. However, because of the lack of TCPs retransmission mechanisms, packet loss is more likely using CoAP and with low packet loss MQTT can experience lower delays than CoAP\cite{karagiannis2015survey}.

\subsection{AMQP}
The Advanced Message Queuing Protocol (AMQP) is ''an open standard for passing business messages between applications or organizations''\cite{amqp}. AMQP originates from the financial industry has been reported to have an environment of 2000 user processing 300 million messages every day. AMQP supply asynchronous publish/subscribe communication and has a ''store-and-forward'' feature that makes it reliable even after network disruptions \cite{karagiannis2015survey}. 
\\\\
IoT is still an emerging concept and therefore there are a lot of competing standards trying to solve the same scenario. We see that cloud providers have support for a all of these standards in their IoT platforms.

\subsection{zigbee}

\chapter{Design}

Goal of the design
Discussion of design choices
Design details

\section{Describing an ideal system design}
\begin{figure}[h]
\caption{Refrence system design}
\centering
\includegraphics[width=8cm]{ideal_system_design.png}

\end{figure}
\section{Golden Design}
Early on we had a clear vision on how to build a system that scales well when the amount of traffic. That can handle multiple user sessions without impacting the performance of the data gathering APIs. Figure 3.1 is an illustration of the reference design that we started building our application based on. This diagram omits any gateway or edge node in favour of a system where we assume that we have good networking coverege.
\subsection{Load balancer design}
A load-balancer would be responsible for unifying all of the stateless apis and exposing them as a single DNS address. As an example we could have a website like example.com point to a single machine/ip. Or we could have the same address point to a load balancing server/service that divides the load between a number of the stateless api. A load balancer typically works on a algorithm to assign requests to any of the apis in the system. A common example of this type of algorithm is round robin. This would be a good strategy for handeling load increases as more and more instances of the stateless apis can be added to the load balancers list of available workers.
\subsection{Api design}
We assumed an api first approach where we need to think about the resources that we will need in the api. As an example we approach the api design by breaking down the model of the objects that would need to be contained within our system as this quickly can give a good impression of the resources that we will need to create, read, update and delete. Generally web-api design has a clear segregation between resources by implementing controllers that have very specific jobs. This is a design pattern that lends it self to document based storage solutions as the objects and resources that are accessible through the api, are typically very similar to the schema of the database.
\subsection{Database design}
The database design is often based around how we choose to structure the data. And in our case we have a clear definition of what data points we want to gather and what the model for said objects would be. One thing to note is that through this project we always thought of our model as one that is a natural candidate for nested objects instead of a relational database. Working with document databases has a range of advantages that we knew about during our design phase. The implementation idea was to have a high performance implementation of MongoDB, and if we needed to we could add long term storage, and high performance caching to the read api. Another important concept of the design is the read only replica of the database to be used for the view portion of our application.
\section{Model design}
\begin{figure}[h]
\caption{Model design}
\centering
\includegraphics[width=8cm]{model_golden.png}
\end{figure}

The model design says something about the relationship between the different objects in our system and how they relate to each other. Here we focus on taking in a view of the whole system, and keep in mind that we would like to build something that can be scaled beyond one single location. The design also takes in to account how the user can drill down from a farm level, or if the route is chosen, how interested parties can drill down in to the data from a region level. We envision that the proposed model design can ease the analysis of aggrigate data, and at the same time provide the foundation for a fast and reliable way to update the database.
\\\\
While document databases are a good way to nest data in a manner that is logical given their relation. It is still important to keep in  mind what parts of the model could potentially slow down an insert if it needs to go through a lot of steps to reach its destination. As an example farm.mote.observation is a good way of finding data when you are in the context of a view. But when we consider the insert, it will be more beneficial to log an observation on the mote level, instead of considering what farm or area the observation belongs to first. This design challenge in the model can be solved by saving references to the different relevant IDs instead of embedding documents. In the API layer there has emerged technologies that abstracts away the embedding and allows the user to select whether they want to see the nested version of an object. GraphQL allows users to submit a schema that they are interested in, and as long as the api has definitions for where to look for the refrenced objects it can pass back the full embedded document/nested object to the user. This achieves the same positive effects as the embedding, without the drawbacks of intermingling the objects.

\section{Technology}
As computer programmers and engineers we are inclined to think that there is a “tech fix” to all of humanity's problems. And in this paper we will argue that in the case of water usage, it is not just blind technology optimism. But a feasible way to help reduce water usage and alleviate some of the stress that is caused on our environment by agriculture. The main factors leading to this being more feasible now is the dropping price of computer processing in the cloud. Which enables us as computer programmers to churn through vast amounts of data and has enabled. And made machine learning much cheaper and more available to the masses. Another contributing factor is the dropping price of the type of hardware that is used in this project, as well as batteries which would help with making the motes wireless.
\\\\
Our hope is that you can identify a model based on some of the measured variables in the system. And eventually remove the need for measuring all together. 
\\\\
Ultimately the goal of this type of system would be to identify a relationship between certain variables.  Given that the field of study we are in is not agriculture we will keep information on this subject. But one can definently theorize that there is a multitude of variables you can look at. Some examples of values that can be read with sensors and cameras are ph-level of the soil, rate of photosytesis (using ir cameras), sunlight hours, temperature, moisture. If such a relationship between important agricultural variables can be established, you can use this type of monitoring system to aggrigate trends on a farm, county or contry level and say something about the optimal way of treating your plants within a region. We also theorized that over time as you identify the most important variables you could in priciple start reducing the number of motes in the system and run predictive analysis on the data sets based on aggregate data for a region. One such factor could be weather forcast to analyse and predict the watering strategy that should be implemented for a farm, county or region. 


\section{Off the shelf hardware and software}
In the world of computer hardware and software we tend to see leapfrogging in technology which yields great results. One such revolution has happened within sensors and sensor networks, is the high availability of general purpose, small computers like the raspberry pi or the arduino. As well as the falling price of fit to purpose sensors like the humidity sensors we have utilized in our proof of concept. This enables researchers and entrepenours to rapidly prototype and test hypothesis with harware that used to cost hundreds of thousands for a fraction of the cost. This has fueled a plethora of easily configurable software and hardware. And spawned a sub culture of what we call precision farming enthusiasts online. 
\\\\
When it comes to the commercial side of this type of technology, this market seem to have gotten a lot more mature recently. And there is a lot of content about this type of technology as well as commercial implementations of similar technology.
\\\\
There are examples of commercial operators in this space and in this section we will have a look at some of the operators in this space
\\\\
"Inmtn" is an industrial company focusing on among other things irrigation control. They have industrial scale solutions which can be used for monitoring the factors that we where able to monitor in our proof of concept. This company utilizes a more generic type of control architecture called SCADA. This type of architecture focuses on the entire production process instead of only the gathering of information like we have focused on in our proof of concept.
\\\\
Schneider Scada implementation
\\\\
Wildeye is a company that seemingly focus on smaller operations as one of their key selling points is the price of their hardware and software. Their solution looks a lot like what we have outlined in our proposal. They have weather proof sensors which are wired to a central computer which distributes the information wirelessly. This differs from our proposal as you are not running a wireless sensors alone, but in tandem with other sensors. 
\\\\

\section{read write challanges (bandwith)}

Any system that is processing large amounts of data will need to be able to process these data as they arrive. In our scenario we had five motes sending out data every 15-30 minutes. But we still spent time thinking about wether or not our approach is scalable. We will talk about api scaling with different techniques as there are many ways to achieve good performance while still keeping in mind the cost of a system. As an example if we where to see that the api is not able to handle a high amount of concurrent requests we could deploy multiple copies of the API we developed behind some load balancing mechanism to achieve good horisontal scaling. Apart from this we also need to consider the following. How does our selected api middleware deal with a high number of concurrent requests. How does our selected programming language deal with asynchrounous requests.
\\\\
We also have major read write challanges when it comes to transmitting our data. Do we want to transmit directly, or would it be more benefitial to group data in an edge node and transmit larger chunks of data. These are important things to consider as event replayability and reciliancy in the network is important to give a clear picture of the water usage in the monitored area. This will be covered more in the architecture and future work portions of the text.

\section{data stream management}
There are a number of things to consider when we look at the streaming system that is set up for this system. The implementation is quite rudamentary, but there are a lot of functions that can be added in the future. The main question one needs to answer when setting up a stream management system is whether or not you want to run queries on the live stream. In our case if we where to do a full implementation with analytics, notifications and real time visualization. One would need the ability to query the stream. This can be achieved in many ways but one suggested approach is to use the offerings from the big cloud vendors. Iot gateway can feed data to a stream analytics service in azure and allows for real time querying of the data before it hits the database.
\\\\
Given that we did not implement any systems like this one could still use a pub/sub pattern to subscribe to arrival data in the database. There are many frameworks which allows for notifying listeners when a new object arrives in the database. We experimented with facebooks graphQL spec implemented in node which allows for subscriptions to sockets that publish when new data arrives. This would not yield true real time analysis, howevere given the requirements described in the reference architecture one can imagine that true real time is not needed as the data we monitor does not have a tendency to change rapidly.
\section{hosting}
The way to host is a big cost driver and to a large extent decides the feasibility for this type of set up in large parts of the world. And unfortunately we are seeing that the cloud capabilities in areas where water is the most scarce is not the best places to utilize cloud technologies. We have found that the system we built also can be hosted as an on premise application on a single network. In that case the cost of the system would be different, and could vary greatly in implemenatation cost based on the number of motes deployed.
\\\\
When considering a cloud deployment of this system the most important things to keep in mind is the network coverage. Which over most of Norway is good, this system could also be deployed using the 4g network as the data packet sizes are very limited. We can also cache the data from the motes on an edge node which can act as a gateway to the cloud, or in certain cases cache the data on the mote itself given that we use hardware with substantial amounts of ram ane flash storage. For reference the esp8266 used for this project usually ships with 4MB of flash storage. 
\section{power consumption}
We learned early on the the power consumption of the mote is hughely important and undoubtedly the biggest challenge while we where developing the motes used in this project. The longest life we where able to get using the "all in one" versions of the esp8266. But found that the unit is simply consuming to much powe with its additional modules. We started to go down the path of trying to design our own low power implementation, but ultimately we decided that this was not the main focus of our efforts. Howerver there are a lot of things that can be done from a technology perspective. And as you will be able to see in the attached code we have used  power saving functions that are built in to the "nodemcu" operating system that is loaded to the esp8266.
\section{data integrity}
There are many levels of integrity when it comes to data within a sensor network. In our opinion the primary factor for data analysis is the overall integrity of the data-set, as this is what will be used for analysis of water consumption. That is why the main challenge for this system and the main consideration for our system is whether a cloud based database can support the integrity we will need to gain insight through using our data-set. 
\\\\
The secondary factor of data integrity relates to the connectivity of our sensors, and whether or not we are able to transmit precise and timely data for out monitored variables. In this case this has way more to do with the networking of our system and the way we program our motes. This will be covered in detail in the concrete implementation of this system.
\subsection{eventual consistency}
The concept of eventual consistency would be followed for this type of system as you might end up with a dataset distributed across multiple regions and a read and write specific database. Meaning that we would not necessarily need a always consistent model given that outliers would be weeded out if we normalize the data and that the event occurring between two data points would be from an intuitive standpoind relatively trivial to approximate.
\section{scalability}
Any system gathering large amounts of data would need to be easily scalable. There are a number of strategies that will be covered in the implementation notes, both for the concrete implementation and the reference architecture that we will present. One important concept is to keep state away from any interface that interacts with the database, and leverage the timestamps to do the heavy lifting of stitching the data back toghether. We will cover loadbalancing and stateless apis in the section on how to achieve acalability in a rest api.

\section{An introduction to a proposed implementation}
Our proposed solution consists of a simulated Wireless Sensor Network(WSN), a real life small scale WSN and a cloud hosted web application. The simulated WSN is a representation of a general network of sensor nodes measuring soil data in an urban agricultural setting. The simulated WSN is based on data from a real life small scale WSN we implemented. This real life WSN consisted of nodes based on the NodeMcu development board which collected soil data our houseplants. The data was stored on a web server and later used to create datasets for the simulation. The WSN was active and collected data for several months. 
\\\\
The goal of the simulation is to test what kind of hosting service is nessecary.
\\\\
The solution is a testement to how minimalistic the application and interface part of this kind of system can be done. We host our application on a very affordable cloud service to insignificant cost and show that this kind of system can easily handle the kind of traffic a WSN produces. Based on datasets we produced in 2017 we can scale the WSN to stress test the web application. We will now talk about the data collection phase and explain our solution in greater detail.

\chapter{State of the art and refernce projects}

\section{Applications in agriculture}
\subsection{Automated Irrigation System}
J. Gutiérrez \textit{et al}. \cite{6582678} has developed an automated irrigation system using a WSN comprised of soil-moisture and temperature sensors placed in the root zone of the plants. The systems consists of senor units, gateway nodes, the irrigation mechanisms and a web server hosting a web application. The sensor units communicated with the gateway nodes using ZigBee and the gateway nodes transmitted the data to the web server through a GPRS module via the public mobile network. The sensor units was powered by both battery power and solar power and contained the two sensors, a microcontroller unit and a ZigBee transmission unit. The gateway nodes contained a ZigBee module, a microcontroller and a GPRS module. The gateway nodes received, identified, recorded and analyzed the sensor data as well as controlling the pumps in the irrigation system.
\\\\
The motivation for the system was to optimize water use for agricultural crops in water scarce environments. During the 136 day test in a sage crop field the ??? showed ''water savings of up to 90\% compared with traditional irrigation practices of the agricultural zone''\cite{6582678174}. J. Gutiérrez \textit{et al}. argues that ''the modular configuration of the automated irrigation system allows it to be scaled up for larger greenhouses or open fields''\cite{6582678166} and that the system is ''feasible and cost effective for optimizing water resources for agricultural production.''\cite{6582678174}. Figure \ref{AIS1} depicts the proposed architecture.

\subsection{Vineyards}
J. Burrell \textit{et al}.\cite{1269130} focuses on how the agricultural workforce should play a role in the shaping of WSNs. The goal is to develop a WSN for use in a vineyard. The author has a human-centric research approach and use ''ethnographic methods including interviews, site tours, and observational work to broadly understand the work activities and priorities of the various roles working in a vineyard''\cite{126913038}.

\subsection{On-farm frost monitoring}
F.J. Pierce \textit{et al}. \cite{PIERCE200832} describe ''hardware and software components of technologies developed for regional and on-farm sensor networks and their implementation in two agricultural applications in Washington State, an agricultural weather network and an on-farm frost monitoring network''. The objective of the work was to use the technologies to improve ''important farming operations that add value through improved efficiency and efficacy of targeted management practices''\cite{PIERCE200832}. For the on-farm frost monitoring network the authors implement a radio based star topology telemetry network and a ad-hoc software solution to collect, manage and display the data. The authors cooperate with the farmers to determine the requirements of the system. The system needs to ''report air temperature every minute to a computer operating anywhere on the farm''\cite{PIERCE200832}, it was important that ''multiple workers involved in the frost protection event be able to view current and trend data for each monitoring station in real time''\cite{PIERCE200832} and that ''The user needed to be able to set temperature thresholdsthat would trigger an alarm at their computer''\cite{PIERCE200832}. Although meeting some problems with weather and the radio communications, the WSN helped the farmers make more informed decisions about what to do when the frost occurred. 

\subsection{Aquaponics}
P.C.P De Silva \textit{et al}. \cite{7780266} has applied an IoT architecture to water quality control in an aquaponics system. Aquaponics is a soil less growing method in which fertelizer is created from within the system boundaries using fish waste. In this context the food production process is treated as ''a distributed industrial manufacturing process with strategic data acquisition, automated optimal control, and automated process control which is to be implemented in urban and rural areas.''\cite{77802661}. The system is comprised sensor and controlling units denoted as endpoints, an an IoT cluster, a  streaming analytics server and an data analytics server. The system is intended to facilitate for urban agriculture and addresses food security issues for crops produced intensively in limited spaces.  

\section{Off the shelf hardware}
There are not a lot of off the shelf hardware in so far as ready made packages. Most of the moisture sensors that are commercially available are manual read out sensors that tell you when to monitor. The same sensors that where used for this project is in my experience what comes up when searching for this type of hardware. There are however different types of this sensor that have other capabilities than the ones used in the proof of concept. 
\section{off the shelf software}

\section{exploring dashboard offerings}
\subsection{tableu}

Estimation of excess water use in irrigated agriculture: A Data Envelopment Analysis approach

\chapter{Cost constraints}
\section{Cost and context}
Without getting to philosophical we need to consider the simple question: What is expensive? This is important as it sets the context for the discussions that will follow in this chapter. We will be looking at costs in two different ways, what is the cost to society, and what is the cost measured in monetary value. Focusing mostly on the latter. 
\\\\
It is clear in my mind that excess water usage has negative externalities and costs to our society. Consider the glaring examples of sink holes and depleted aquifers. Although edge cases to a certain extent now, we still need to consider the negative effects of excess water usage, this is mainly to understand why reducing water usage in agriculture is important (kanskje finne noe om hva som bruker mye vann). Depleting groundwater can also increase inequality as it can quickly turn in to an arms race where farmers drill deeper and deeper looking for water.
\section{Considering the world wide availability of the internet}
There is a discrepancy in the availability of internet connectivity in the developed and the developing world. And although this is improving over time with adoption of 4g/5g and investment in internet and communication infrastructure. This is one of the major hurdles that could face implementation of a WSN system in certain regions of the world. And unfortunately the areas that are especially drought prone can also have other challenges on the infrastructure side. The solution we have presented assumes a certain level of development within telecommunications and internet infrastructure that is not available through the entire world.
\section{Cost of hardware}
Make a table including the components we used and the price

\section{Cost of hosting}
The cost of hosting is very interesting at it is highly linked to the size of the wsn that is being deployed



\chapter{Implementation and evaluation, An overview of a proposed implementation}

Sometimes split into two chapters depending on complexity etc.
(Discussion and) Description of implementation choices made 
Evaluaation: description of the experiemnts taiming to show the properties 
and performace of what you implemented in relation to the goals (how do you show that you have reached the golas of the thesis)

\section{Implementation conseptual}
Proposed architechture for large scale agricultural monitoring
Cloud technology has enabled us to set up large scale systems at a much lower upfront cost, this is very advantageous when it comes to monitoring and ingesting large amounts of data. We would like to propose some architectures that we have found efficient and is a good fit for our application. 
\\\\
There are three main components that need to be considered and the first and most important is how we link our motes to the cloud. There are a multitude of concerns that need to be addressed like potential bottlenecks, security, and choosing the correct gateway for our motes.
\\\\
We have chosen to go with an architecture that allows the motes in the system to be highly autonomous/ low coupling and no dependency to each other. This is an choice that requires a bit more bandwidth usage, and is most advantageous when there are no caps, and the internet connection is not mobile..
\\\\
The point is to have every mote operating individually and having an isolated connection to our ingestion (platform/engine). This means that the motes have their own connection to a router and is free to communicate with the internet. This raises some challenges regarding security as a larger part of our sensor network is internet facing. This is something that would have to be treated carefully and given more thought in a production environment. For the purposes of this paper we will provide a brief overview of iot security. This layer of the architecture also describes the network (layout/topology) and the structure that we propose for a multi site setup for precision farming as well as the network structure for our (demo/trial) environment.
\\\\
Each mote is connected to the internet via a normal wifi router. Every mote has an ip address, but you are not able to give our test kit remote instructions. Changes have to be made by flashing new software to the chip. Our motes acts just like any client on the internet and is able to send post/get requests just like a normal computer. This means that it is very easy to program the motes for any developer with basic web programming knowledge. The computing power requirements for the motes are very low as the system we have designed is intended to read data at a low requency. (test data collected at 30m intervals).
\\\\
The mote network fits the description of a star network where the common connection point is the router that gives access to the internet. There are multiple configurations that are relevant in an iot network, and star is one of the more rudimentary topologies. But in our case this is what makes the trial environment easy to set up.
\\\\
Ingestion is a very important component given that it would need to scale when more monitoring sites are introduced to the system. We have designed a simple ingestion engine to compare with cloud offerings from multiple providers. This ingestion engine is described in more detail in chapter (111). The ingestion is a one to many relationship and the data has a single point of entry. This is the same in both azure and our custom ingestion engine. There are a lot of things to consider when creating the infrastructure to handle vast amounts of data, and the entire stack of the technology needs to be optimized in some way. A lot of the work in assessing the feasibility of iot in agriculture is selecting the right transfer protocols for optimal power consumption.
\\\\
 When the data has traveled from the “ground” to the cloud we are dependent on reliable and cost effective storage of the data. Given that a lot of iot devices uses json to communicate with a server a natural choice for storing data is a json based document database like mongodb. This type of database is capable of a lot of read and writes per second. Due to the nature of our application write performance is the most important. One big advantage of cloud based data base systems is that they often do automatic replication over multiple locations something witch is great for data redundancy.
 
 \section{Site deployment conceptual}
 \begin{figure}[h]
\caption{Site deployment}
\centering
\includegraphics[width=10cm]{hjemly.png}
\end{figure}

 Following is a conceptual site deployment for monitoring a grass field of approximately 11ha, which incidentally happens to be the grass field of my farm in Hedmark. The idea is to think of this farm as having to "fields" and both fields being of interest. Why do we want to split these fields? We have observed that the yield is significantly different as one of these fields get a lot more sunlight than the other. It will also be interesting to describe how we theorize the networks requirements over such a spread out area.
 \\\\
 Field A is approximately 8ha while field B is approximately 1.5ha. To cover this entire area with wireless signal you will need to set up a network that can hand over connectivity in a seamless manner. We could go many different routes for the configuration. But we theorize that the easiest way to set it up is using repeaters for the signals. We would probably need to run power to some strategic locations in the field. Assuming a modest range of 150m we could in theory cover the field by using two WIFI access points, and with amplification we could cover both fields. Just from this exercise we see that networking could be a hurdle, but we are certain that this would not be a show stopper for gathering data as there are alternate ways of communicating via radio signal that we have not explored.
 \\\\
 When looking at the placement of the sensors we would consider the elevation, and the sun conditions in the patch we are deploying the sensors as seeing how much the reading vary would be hugely beneficial as it could tell us a lot about the amount of sensors we would need to deploy to get satisfactory readings. As well as some of the different areas of the field being more prone to retaining water and having different soil qualities. 
  
\section{Implementation}
\section{Hardware implementation}
The harware implementation was by far the most challanging part of our work and we burned through a lot of time exploring different options for the begining of our work. However it was the interest in exploring sensors and the link between computer software and the real world that sparked the topic of discussion for this project.I will lay out a short timeline on what we tried and then conclude this section with how we ended up setting it up as well as some information on future development that could have been done to create a long lasting sensor package that could be deployed 
\subsection{Summer 2016}
As a hobby project during my final year of college i started looking a lot at the raspberry pie and the GPIO (general purpose input output) pins of the microcontroller and the different use cases. I had become inspired by a smart plant project i found on a Microsoft blog to try and create something similar myself. The first part was finding a low power microcontroller that had similar capabilities to the Raspberry pie and could be used with the same sensors. This is the time i learned about the NodeMcu implementation of the esp8266 chip that has been mentioned throughout this text. Substantial time went in to researching the different options that came along with the microcontroller. There are different micro-os'es available for the nodemcu and i settled on implementing the one that is bundled with the product. There is also a micropython implementation avaialble but i found the community resources around this os implementation to be smaller than the lua based nodemcu package that came pre loaded.
\subsection{Fall 2016}
In the fall of 2016 a the majority of the time was spent tweaking the package deployed to the nodemcu. The way the OS is built up is by including different packages. As an example to http capabilities would not work without including the http/wifi packages. To add on to that you can not choose to include all the packages as it can result in a build that takes up to much of the internal memory and results in an unstable os deployment. We also had a major problem where the mode of the gpio pin used to read the moisture level on the resistance based sensor. This turned out to be a mode flag for the ADC method that reads the pin value, and could be fixed by modifying the binary deployment of the OS. During this time we also experimented a lot with different enclosures and battery configurations, but ultimately decided we where better of continuing the project with usb powered motes given that the max lifespan we where able to achieve on battery was approximately 7 days.
\\\\
The cause of these battery problems was that the power regulator that was bundled on our all in one implementations of the esp8266 was drawing to much power, and we would have to build our own low power implementation to get this working to a satisfactory level.
\subsection{Spring 2018}
Testing of new sensors indoors happened through 2018, this was a new way of reading data as the new sensors we ordered used a digital system for communicating its values. This was where the STH11 library was added and we where very lucky that somebody had already implemented what we wanted to do on the NodeMcu. As this sensor is most widely used with Arduino boards and has better community resources built up around it. Ultimately we found that we where burning through far to much time focusing on the hardware implementation, and decided to focus more on refining the architecture and testing the system through the mocking described earlier in the text. Although it was costly both in time and money, we learned a lot by implementing this sensor, and we also got good readings on the temperature of the soil. However we where never able to calibrate the moisture sensors to a satisfactory level, and decided to return to the resistance based sensors. While still recording the soil temperature using the sth11 sensor. 
\subsection{Conclusion on hardware implementation}
Ultimately i think we made some mistakes in getting too focused on the hardware portion of this project given that this was not the primary subject of research. However it lay the foundation for the answers that are given in this text given that we actually implemented the concepts that are discussed. The implementation gave us a lot of insight to the costs of deploying such a system. And is an important factor in our conclusion and answers to the questions posed by this text.
\section{Software implementation sensor packages}
Before you can start developing on the nodemcu firmware of the esp8266 there are a number of requirements that needs to be met.
\subsection{Building the firmware}
The team that maintains the nodemcu firmware also develops tools for flashing the firmware to the esp8266. This tools can be downloaded and used with relative ease from both widows and unix based systems. (we used windows) They also offer cloud builders that will include the packages that you will need for developing your desired functionality. This is through an intuitive UI and they will acrually email you a finished binary version of your firmware which you can in turn flash on to your chip. All of these offerings are well documented and well maintained. There has been a number of updates since we started experimenting with the software back in 2016.
\subsection{Building the software}
When the firmware is loaded, you can communicate with the Nodemcu via usb as the all in one implementaion of the esp8266 we used has a usb to serial chip. There is a IDE you can use to send code to the board to be executed live, or you can upload files. This is very similar to an interactive terminal like a developer would be used to in something like python. We primarily developed by using the IDE for prototyping and then implementing the same code as a file to be run by the init.lua file. (see reference in chapt 7.1)
\subsection{wifi}
One of the drawbacks of this style of development is that you would physically have to flash the motes when something in the system changes. As an example if the name or password of a wifi connection where to change the motes would have to be flashed again. This is why in a productuin implemenatation we would argue for a dedicated wifi network dedicated to this sensor network. The code for connecting to the wifi is one of the steps we will retry without putting the sensor to sleep, as this is vital for communicating the data point of the sensor.
\subsection{Reding data}
We had two different versions of the main.lua code for reading the moisture of the sensor. One for the initial sensor that we used (resistance based) where we leveraged the built in functions for reading the resistance of the adc pin of the gpio of the nodemcu board. And the second implementation leveraged the sht11 library (see reference in 7.2). We found that both of these read methods worked consistently.
\section{Software implementation server}
The server we wrote for this implementation was a highly simplified version of what would be needed in a production environment as i was lacking IAM, authentication, logging. We chose Node as it has a good track record when it comes to developing restful apis as well as both project members having good experience with the language. There are also a wealth of cummunity resources and libraries available for the language. The reason we chose to implement it like this is that it is relatively trivial to implement a statless-api which is critical to achieve good horizontal scaling of the edge resources of the proposed data collection platform.
\section{Software implementation frontend}
The frontend implementation is all about ease of use, the climate has changed quite a lot since 2016 with react coming out as the dominant framework for front end development. However the arcitechture of this application allows us to quickly change out the front end, so even though the front end implementation is quite outdated today, we could easily re-use the same api calls and javasscript packages used for displaying the data. This is also something we gave a lot of thought during the development of this project and is why we are such big fans of an API approach to development. 

 \begin{figure}[h]
\caption{Water consumption graph}
\centering
\includegraphics[width=12cm]{watergraph.png}
\end{figure}

\section{Cloud deployment}
We explored a range of options for cloud, and it is covered extensively in the cost constraint portion of the text. We had multiple iterations of the cloud deployment. The most like a continuous deployment/ continuous integration we came was when we implemented the front-end on azure using azure dev-ops and visual studio team services to create a hook in to our github library. However our actual long running implementation focused around much more manual processes on Digitalocean. We had both the database, server code and front end being served from a ubuntu server. This allowed for a lot of flexibility as well as a very low cost per month (5usd + vat). We did test hosted apps like "meteorDB" and azure web apps, but ultimately found that running our own linux server was easier for a proof of concept project.
\section{Site deployment}
 \begin{figure}[h]
\caption{Site deployment 3. iteration}
\centering
\includegraphics[width=10cm]{mcu+sensor.jpg}
\end{figure}

When we implemented the proof of concept we started off with an outdoor deployment in a greenhouse as the original idea was to also control irrigation using a threshold that would be discovered by the data. Because of this we had additional complonents in the system like a motor controller to run a pump as well as a sonic sensor to monitor the level of our tank. Although this was an interesting execise in itself we quickly relized that although fun, cortrolling the full system would give limited value to our porject. This is because through our research we started to see that the area that was lacking insight was the data on how the watering should be done. We therefore shifted focus away from the sensor housing, running pumps and building out plumbing. And instead decided to scale the project back a bit and focus on an indoor deployment. This made a few things way more simple. We no longer had to worry about water proofing the sensors. It allowed us to abandon the work on making the sensor battery powered. This allowed us to focus more on the tech platform and the architecture of the application which was more in line with the subject of research. 
\\\\
It is worth noting that we where able to run the system in the greenhouse for up to five days on battery power, and we started to see some of the challanges that would face the sensors outside in the field. One example of unexpected observation was when we noticed false readings when the sensor got heated by sunlight. This is one of the reasons we decided to purchase one of the more advanced sth11 sensors as this problem would be eliminated by the sensor reading measurements in a different way.
\\\\
In our final implementation we had the same type of sensor package deployed to five indoor plants. We collected data for 2-3 months in the fall of 2018 and this data created the grounds for the simulation program that we wrote to stress test the application.


\section{WSN notes}
Wireless Sensor Networks(WSNs) are “a network of battery-powered sensors interconnected through wireless medium and are typically deployed to serve a specific application purpose”[1, p68]. This type of network are applied for a variety of different purposes, including environmental monitoring, machine monitoring, safety management, smart buildings, and for many other purposes. For any application this type of network would be comprised of a variety of sensors, such as temperature, humidity and volatile compound detection.[3] Technological advances have made small, low cost, low power and highly customizable devices commercially available and has improved the viability of large scale sensing networks with a large number of intelligent sensor nodes. A sensor node, nicknamed “mote”, in a WSN typically equipped with one or more sensors, a sensor interface, processing units, transceiver unit and power supply [2]. 
\\\\
For many years the trend in WSN topology has been a network comprised of several motes and one or several gateway nodes. In this type of network the motes communicate with each other and the gateway node using a radio frequency communication technology like Zigbee or Bluetooth. The gateway node can have several functions including computation and internet connectivity. The motes in this type of network can not communicate with the internet directly and relies on the gateway nodes for this making the network a closed or proprietary system with limited communication to the external world. Current trends are pushing WSN in the direction of the Internet of Things(IoT) using the Internet Protocol(IP) to connect every mote to the internet [4]. Connecting individual motes to the internet using  IEEE 802.11(WiFi) enables the utilization of IoT platforms and other resources in Cloud services. In some cases applying a gateway based architecture to a IP-enabled, WiFi connected WSN can be the natural choice, especially if the gateway node(s) are able to perform challenging computing tasks. This architecture follows the concepts of Fog Computing(or Edge computing). In Fog Computing... If your system demands both low latency and computation on large amounts of aggregated data, a Fog architecture would be a natural choice. 
\\\\
Motes can be arranged in different ways. Motes can send traffic through a gateway or a sink node which forwards it to the internet, or if the motes are connected to wifi, they can send the information directly to the internet themselves. Your network topology will be decided by the the sensing situation. If the motes are deployed in a controlled environment with short distances e.g. inside a building, where a Wireless Local Area Network(WLAN) is possible the motes can be connected to the internet with wifi. If the motes are covering too large of an area or for some reason its not viable to use a WLAN, the motes can be connected to an edge/gateway/sink node using radio communication like ZigBee. If the distances are too great one can do a distributed zigbee network or have nodes communicate using GPRS.
\\\\
WSNs can be seen as a part of the Internet of Things(IoT).
\\\\
The main difference between a WSN and a RFID system is that RFID devices have no cooperative capabilities, while WSN allow different network topologies and multihop communication.[3]
\section{tcp vs udp}
For applications where it is important that you get all the info for example products that if faulty will result in cars crashing and people getting run over. It is very important to ensure that your data link is reliable you should never use udp. Udp is a protocol that is inherently unreliable and should be used for applications where data loss is not an issue. In a grid of traffic lights, for example in an intersection, it is very important that all the traffic lights have the right colour at the right time.
\\\\
“When a UDP message is sent, it cannot be known if it will reach its destination; it could get lost along the way. There is no concept of acknowledgment, retransmission, or timeout.”
\\\\
TCP is a connection based transmission protocol, so it is more reliable for applications where you need to be sure that the data gets to the target it is supposed to. The drawbacks with tcp is that it is more heavyweight (more data is exchanged) than udp, but for simple applications like this we feel that it does not really matter.
\\\\
This is why we have concluded that we would like to keep tcp; it is relieable, and that is exactly what you would need for a real world trafick light. And a altough this is a simulated traffic light we find that we learn more when we try to keep our projects real world oriented.


\chapter{Conclusions and future work}

Summary of main observation, key results and conclusions, limitations of the work, suggestions for future work

\section{Overview of questions answered}
In this text we have explored the concepts and technologies that make up Cloud Computing and Internet of Things, we have explored in depth the configuration of Wireless Sensor Networks and we have given examples of how these technologies can be applied in agricultural scenarios. We see from these examples that applying the knowledge of agriculturalists is crucial to the success of the efforts and that the knowledge derived from the systems help the users make the right decisions. We want to further explore how to make these technologies more efficient, how they can become more accessible to users and how they can be applied to emerging agricultural techniques like Urban Farming, Hydroponics, Aquaponics and Vertical Farming. Primarily we are interested in implementing this in a norwegian context as we have other challanges emerging with long droughts during summer in recent years and very wet periods during fall. Our intuition tells us that a lot can be done to increase the yield of livestock feed crops like grass. Which has proven highly important in a Norwegian context
\section{future work}
Fokuser på videre arbeid, og ikke bland det med arbeidet som alt er gjort
\\\\
Something that became appareant throughout the project was that out initial assumptions regarding what factors to monitor where too limited to give  a overall picture of the health of a major farming operation. Something as basic as what depth to put the sensors at can become a  major study in itself and I would have benefitted a lot from research and discovery around this. Although we ded not go to deep in to the plant science part of this project we have learned that there are many factors that contribute to the state of a plant. And we have also learned that there are many factors that impact water retention in soil. Is the soil sandy, is it hard packed, what level of fertalizer is present. Learning more about the different factors, and monitoring how the different variables impact our reading would have been very interesting.
\\\\
It is also natural to think that we would want to do more with the platform we have started to develop. A few of the following items come to mind, security, automation of deployment and user and role engineering. These things where not applicable in a proof of concept scenario. But would have been a must if I ever wanted to  launch this system in a production setting.
\\\\
Mote software deployment is something I immediately saw huge problems with and the deployment model for software would have been a massive problem in a production setting. We had some ideas on how to improve this, but none that where possible to implement with the architecture that we had chosen for our project. Her I would like to invest more time in to the data transfer protocol and the network topology. I think as an example utilizing edge nodes with wifi and running the rest of the nodes off of radio signals could reduce the points in the system where we would need to change configuration and update software. If we had problems with the http protocol implementation or needed to change or update the code, we could target the edge node as all the connectivity in one node at the edge as a gateway to our server. The goal of this suggested further work would be to achieve a eaven greater separation of concerns, having the motes focus on data gathereing and allow the edge nodes acting as gateways to deal with connectivity, data caching, uploading data. As well as limiting the number of nodes in the network that we need to change if we where to update the software or change the technology of the apis at any point in the future.
\\\\
I have alluded to future work regarding data analysis in this text. I definitely believe that there is a lot of room for statistical analysis on the data sets that the sensors generate. This was an area where i would have liked to researched further but due to time constraints it is ending up in future work. My theory is that this data can be fed in to a machine learning model and using factors such as sun hours and weather forecasts we should be able to say something about the rate of consumption and evaporation. Resulting in more accurate and precise irrigation use, thus contributing to reduced water usage and ensuring limitation of excessive irrigation.

\subsection{management of sensors}
begrensingger i vårt arbeid, hva er mulighetene i annen metoder

\chapter{notes to be incorperated}

Jorbruket er et av de mest vanninntensive næringene vi har. Jeg ønsker å undersøke mulighetene for å bruke informasjonsteknologi til å redusere bruken av vann i jordbruket. Dette mener jeg er mulig å oppnå gjennom å bruke relativt enkle og billige sensorer i samspill  med cloud computing plattformene som er tilgjengelig i markedet. Gjennom vårt arbeid med denne hypotesen er vi kommet over flere artikler som underbygger at prosesser og vanningskontroll kan være viktigere enn måten du vanner på.  dette er noe som gjør oss ekstra interesserte i å bygge ut en enkel og kostnadseffektiv måte å redusere vannbruk i jordbruksproduksjon.
\\\\
I denne artikkelen kommer jeg til å gå gjennom grunnleggende prinsipper om hvert av leddene i “stacken”, legge frem et forslag til arkitektur, sammenlikne med eksisterende løsninger og prosjekter og trekke en konklusjon om hva jeg mener er mulig å få til med dagens teknologi. Løsningen jeg vil presentere er bygget på et arbeid med sensor og sensordata i skyplattformene, digital ocean, aws og azure. Jeg kommer til å dekke fundamentene om teknologien, sensordataen, datalagring og cloud for å gi leseren tilstrekkelig grunnlag for å sette seg inn i løsningen som er foreslått i denne teksten.
\\\\
 Jeg vil også beskrive grunnleggende statistikk prinsipper brukt i forslagene for modellering og estimering av forbruk.  plattformen vil gi et sant bilde av hvordan vannforbruket til en gård er til enhver tid.
\\\\
 jeg kommer også til å se på andre faktorer som kan ha en dokumentert effekt på effektiviteten av vannforbruket.  det vil beskrives løsninger for Hvordan disse data punktene kan inkorporeres i plattformen.
\\\\
Det er flere eksempler på prosjekter og produkter som har vist og forsøkt at teknologiforslaget jeg legger frem kan være mulig å gjennomføre. Jeg kommer derfor også til å gå gjennom hvilke løsninger og teknologier som er på markedet og relevante prosjekter innenfor samme tema.
\\\\
I den automatiske verden kan noen ting går under radaren, spørsmålet jeg ønsker å stille er, hvor mye vann trenger egentlig en plante for å vokse? Vanner vi for mye i forhold til det absolutte behov, og hvikle sannheter er det jordbruket globalt har kommet frem til. Kunne vi vannet gresset på Osen eller i Dubai mindre. Hva er verdien av det vannet vi sparer, kan vi forhindre at sprøytemidler og annet avfall fra jordbruket kommer ut i et økosystem der det ikke hører hjemme? Jeg er kommet frem til at svaret på dette er ja. Og gjennom denne teksten kommer jeg til å vise deg hvordan vi kan gjøre det. 
\\\\
Å beskrive en mulig løsning vil også innebære å komme med forslag til en modell som over tid kan si noe om hvordan fordampning foregår i forskjellige klimaer. Det finnes en rekke artikler som beskriver formler/algoritmer for hvordan man skal estimere vannforbruk. I denne oppgaven kommer jeg til å vise noen eksempler på hvordan vår data kan brukes sammen med machine learning produkter fra microsoft og amazon web services. Dette er under videre arbeid, men også en av de viktige lærdommene vi har tatt med oss fra eksperimenter gjort under veis i skrivingen.
\\\\
Arkitekturforslaget vi har utformet kommer sammen med en rekke forslag til hvilke tjenester som kan brukes fra de forskjellige leverandørene.  vårt forslag baserer seg prosjektarbeid av Meg og Kristoffer gjennom sommeren 2017 i denne løsningen brukte vi flere forskjellige teknologier sammen.  Dette gav oss en god oversikt over hvilke produkter Vi trenger for å utforme denne type system.  Denne løsningen ble produksjon satt sammen med sensorer. disse sensorene målte til fuktighet i jorden.  Det er to type sensor Vi brukte under eksperimentet.  den første typen er en enkel sensor som baserer seg på spenningen gjennom jorden.  den andre sensoren Vi brukte har en spesiell utforming som gjør at vann ikke trenger inn i sensoren men at luft slipper fritt igjennom på denne måten kan en måle luftfuktigheten mens en som ligger under bakken.
\\\\
Det er flere andre forslag til arkitektur som er kommet frem under arbeidet men vi har spart disse til   videre arbeid.  blant annet Kunne det vært fordelaktig å bruke køer men vi har oppnådd et løst koblet system gjennom å bruke api.   er en måte å kommunisere med applikasjoner på gjennom bruk av http Hvilken protokoll du bruker er viktig når du jobber med sensorer som krever lite strøm. I vårt tilfelle var dette ikke en utfordring da vi ikke kjørte sensorene på batteri.
\\\\
 sensorene vi bygget ut for dette prosjektet er basert på en billig og enkelt tilgjengelig microcontroller  mikrokontroller som heter node m c u.   denne mikrokontroller har mulighet for å ta input og output. Dette gjør at vi kan tilkobles de sensorene som er nødvendig. i vårt eksperiment koblet vi kun på fuktighetssensorer.  men man kan følge samme prinsipper for å koble på en rekke andre sensorer. eksempelvis for å måle pH verdi eller andre ting som kan være nyttig i jordbruket.  mikrokontroller programmeres med et språk som heter lua. microcontroller kommer med en rekke forhåndsdefinerte funksjoner som gjør det enkelt å bygge ut sensornettverk.  microcontroller støtter også wifi noe som gjør det enkelt å sende data til cloud. figur under for bilde. mikrokontroller  Vi brukte trenger 3 volt for å virke men denne versjonen av microcontroller bruker for mye strøm til å kjøre på batteri. Dette var noe Vi brukte unødvendig mye tid på og det er noe som er løst av Andre tidligere.  Vi valgte derfor å gå over til å bruke strømtilkobling. 
\\\\
 arkitekturen vår følger en enkel tredeling der Det nederste nivået er datagenereringen.  i datagenereringslaget  blir dataen plukket opp av mikrokontroller og gjort om til json  objekt som kan sendes til web API via http.  web API har en kontrakt som er kjent for microcontroller,  denne kontrakten forteller mikrokontroller om hvilken form dataobjektet  skal ha.  når et datapunkt sendes til web API blir det prosessert  og lagret i en database.  databasen endrer ikke på hvordan data objektet ser ut,   det den beholder datapunkt  i samme  form som kontrakten tilsier.   web API tilgjengeliggjør også dataen for vår applikasjon som både kan vise og prosessere data vi kommer mer tilbake til hvordan denne applikasjonen er utformet og hvilke Cloud Services den bruker.
\\\\
microcontroller genererer datapunkter sendt til et web-api. Det er en kjent kontrakt mellom mikrokontroller og web API.  se eksempel på data kontrakt på figur under.  hver microcontroller  sender sine individuelle datapunkter til web API gjennom http. Det finnes en rekke alternativer for å kommunisere fra  microcontroller. disse protokollene kommer vi til å nevne mer om i delen som omhandler wireless sensor networks. for våre formål  var http tilstrekkelig.  det er viktig å merke seg at fra et ressursperspektiv er ikke http det mest effektive. http krever mer data enn andre protokoller. her kunne vi ha tatt andre valg i arkitekturen som kunne gjort det  mindre ressurskrevende å kommunisere med web API.  dette kommer jeg til å komme tilbake til i fremtidig arbeid.  meldingene mikrokontroller sender til apiet  er det laveste nivået i applikasjonen vår.
\\\\
 i web.api mottar vi i vårt konkrete eksempel data fra ca 5 sensorer. all data fra disse sensorene har samme format.  vi foretar ingen transformasjonen av denne dataen. filosofien bak dette er at vi ønsker å bevare rådata og heller La andre applikasjoner gjøre transformasjonene.  på denne måten er vi sikre på at vi alltid kan finne ut hva det originale datagrunnlaget var.  i web API  kan vi både motta og sende data til andre applikasjoner ved forespørsel. når web.api mottar et nytt datapunkt lagets dette i databasen til applikasjonen Her har vi brukt noe som heter mongodb. kontrakten web.api publiserer er også modellen vi bruker for å lagre objekter i databasen. Dette gjør at alle ledd er kjent med kontrakter og modeller.   web API kan konsumeres av flere forskjellige klienter Det er ikke begrenset til vår applikasjon. 
\\\\
For våre formål var det hensiktsmessig å kunne presentere data på en enkel visuell måte. Vi bygget derfor også et presentasjonslag  i form av en web app.  denne web appen henter data fra  web API og presenterer det for sluttbruker dette ga oss god innsikt i hvordan fuktighet i plantene oppfører seg.

\\\\
Det er flere utfordringer man møter på når man bygger opp sanntidssystemer. Men det er også viktig å huske på at ikke alle punktene er like kritiske dersom du henter data fra for eksempel et helt jorde. Det er for eksempel ingen krise å ha datapunkter som spriker fra normalen da det finnes en rekke strategier for forhindre at disse punktene blir med i utregningen. Det er heller ikke særlig kritisk å miste data for enkelte tider da det primært er trenden i dataen vi er interessert i. En annen utfordring man kan ende opp med å møte i sanntidssystemer er data som ankommer samtidig, her finner vi en av de store fordelene med å bruke web api teknologier. Dette er systemer som er bygget for å håndtere høy grad av concurrency. Tenk for eksempel på en nettside, den kan motta tusenvis av “get” forspørsler i sekundet. Vi unngikk også problemer med konflikter i dataen ved å utnytte at hver av micro controllerene har en unik id vi kan bruke for å identifisere hvor dataen kommer fra. Dette gjør at vi enkelt kan håndtere store mengder data.
\\\\
 the mass of agricultural produce per unit of water consumed and economic water use efficiency as the value of product(s) produced per unit of water volume consumed.
kilde for quote
\\\\
Det er mange faktorer som spiller inn når man skal redusere vannbruk i jordbruk.   definisjon Vi har valgt å bruke er effektivitet beskrevet som Massen av produksjonen Per enhet vann som er blitt  brukt.  det er flere tekster som beskriver effektiviteten av forskjellige vanningssystemer.  vår  teori er at det er mulig å forbedre effektiviteten for vanning i jordbruket ikke kun mekanisk men også gjennom bruk av informasjonsteknologi. Et annet aspekt som er viktig å merke seg er at det å bytte vanningssystem er veldig kapitalkrevende. Vi ønsker å vise at du kan bygge opp et system for å forbedre effektiviteten med hjelp av informasjonsteknologi.  tanken er at mengden av  bønder  som har muligheten til å investere i denne typen system er langt større en de som har mulighet til å investere i nye vanningssystemer. 
\\\\
[Sett inn artikkel fra over]  konkluderer med at det ikke er et vanningssystem som dominerer for gårdene i referanse dataen som gjør det best på konservere vann.  dette forteller oss at det er mye å hente i og ikke kun fokusere på  de Mekaniske aspektene av vanning men fokusere i stor grad på  Når og hvor mye du vanner.  som nevnt tidligere er det også større investeringer knyttet til å bytte måte man vanner på.  dette er enda et argument for at modellering og overvåking med hjelp av sensorer og cloud kan være fordelaktig og  mulig for landbruk.
\\\\
 samsvarende ser de heller ikke noe tydelig forhold blant gårdene som bruker mest vann. Dette kan indikere at det ikke nødvendigvis er noe forhold mellom type vanningssystem og overforbruk av vann.  videre foreslår teksten at det ikke nødvendigvis er  det mekaniske vanning systemet,  men prosessene på en gård som har mest å si.  i bunn og grunn er  løsningsforslaget et verktøy  som skal hjelpe bønder med å ta bedre avgjørelser assistert av sensor data og statistikk. 

\chapter{Glossary}
API - Application programming interface. In the context of this text we have used api and web api interchangably. In this context they both describe a way of connecting/interfacing with an application or platform
\\\\
WSN - Wireless sensor network
\\\\
IoT - Internet of things, common catch all word for sensors monitoring or interacting with the real world
\\\\



\chapter{Appendix Code}
\linespread{1}

\section{init.lua}
The init .lua file is the file resposible for connecting do the wifi network and running the deep sleep function of the nodemcu.
\lstinputlisting[language=Java]{init.lua}

\section{sth11.lua}
\lstinputlisting[language=Java]{sth11.lua}

\section{main.lua}
main.lua is the file refenced in init.lua and is responsible for reading and sending the moisture data from the sensor
\lstinputlisting[language=Java]{main.lua}

\section{server.js}
Server.js is responsible for recieving and saving the data sent from the motes.
\lstinputlisting[language=Java]{server.js}

\section{index.html}
Index.html is the main file to control the front end
\lstinputlisting[language=html]{index.html}

\section{dashboard.html}
dashboard.html shows the tables and information for the water usage
\lstinputlisting[language=html]{dashboard.html}

\section{pageController.js}
pageController.js contains the controllor for managing the front end
\lstinputlisting[language=Java]{pageController.js}

\section{mote simulator.py}
Python mote simulator
\lstinputlisting[language=Python]{mote_simulator2_0.py}

\section{mote simulator.py v2}
Python mote simulator
\lstinputlisting[language=Python]{mote_simulator.py}

https://nodemcu.readthedocs.io/en/master/
\\\\
\textbf{SYN, SYN-ACK, ACK}
\\\\
Christoffer Gard Osen was my partner on this project for more than two years, but decided to leave UiO at the end of 2018. Without him i would have felt very lost through this project and would like to extend a big thanks and acknowledgment for all the good conversations, discussion and collaboration.



https://journals.plos.org/plosone/article?id=10.1371/journal.pone.0131422
\end{document}
